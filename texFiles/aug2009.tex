\documentclass[11pt]{article}

\usepackage[english]{babel}
\usepackage[utf8]{inputenc}
\usepackage{amsmath}
\usepackage{amsfonts}
\usepackage[ampersand]{easylist}
\usepackage{authblk}
\usepackage{MnSymbol}
\usepackage{graphicx}
\usepackage[colorinlistoftodos]{todonotes}
\usepackage{geometry}
\usepackage{color}
\usepackage{mathtools}

\newcommand{\bs}{\boldsymbol}


\DeclarePairedDelimiter\ceil{\lceil}{\rceil}
\DeclarePairedDelimiter\floor{\lfloor}{\rfloor}

\geometry{letterpaper}
	\setlength{\oddsidemargin}{0cm}
	\setlength{\evensidemargin}{0cm}
	\setlength{\headheight}{0.5cm}
	\setlength{\headsep}{0cm}
	\setlength{\textwidth}{16cm}
	\setlength{\textheight}{21.0cm}
	\baselineskip=24pt

\title{Applied/Numerical Qualifier Solution: August 2009}

\author{Bennett Clayton}
\affil{Texas A\&M University}
\date{\today}

\begin{document}
\maketitle

{\bf Problem 1.} Consider the following finite element triple:
\begin{itemize}
    \item $K = \text{ a rectangle with vertices } \{a^i\}$, $i = 1, 2, 3, 4$.
    \item $P = Q^3 = \text{span}\{x^i_1x^j_2 \: ; \: i, j= 0, \ldots, 3 \}$. 
    \item $N = \{ p(a^i), p_1(a^i), p_2(a^i), p_{12}(a^i), \:  i= 1,2,3,4 \}$. (Here $p_i$ denotes differentiation with respect to $x_i$).
\end{itemize}

\vskip 1cm

{\bf a.} Show that the above finite element is unisolvent.

\vskip 1cm

\textbf{Solution}: For simplicity, we will work on the unit square $\Hat{K}$ with vertices 
\begin{equation*}
    \{a^i\} = \{(0,0), (1,0), (1,1), (0,1) \}.
\end{equation*}
Working on the reference element is satisfactory due to the affine equivalence of finite elements.
Recall the definition:

\textbf{Definition}: \textit{Let $\mathcal{T}_h$ be a triangulation of $\Omega \subset \mathbb{R}^d$, $K \in \mathcal{T}_h$, $P_K$ a finite dimensional vector space of functions defined on $K$, and $\Sigma_K$ the space of linear forms mapping $P_K$ to $\mathbb{R}$. 
Let further $(\hat{K}, \Hat{P}_{\hat{K}}, \hat{\Sigma}_{\hat{K}})$, be a reference element. 
Then, the finite elements $(K, P_K, \Sigma_K)$, $K \in \mathcal{T}_h$, are said to be affine equivalent to the reference element $(\hat{K}, \Hat{P}_{\hat{K}}, \hat{\Sigma}_{\hat{K}})$, if there exists an invertible affine mapping $F_K : \mathbb{R}^d \to \mathbb{R}^d$ such that for all $K \in \mathcal{T}_h$ 
\begin{align}
    K &= F_K(\hat{K}) \\ 
    P_K = \{ p : K\to\mathbb{R} &\: | \: p = \hat{p} \circ F_K^{-1}, \: \hat{p} \in \hat{P}_{\hat{K}} \} \\
    \Sigma_K = \{ \ell_i : P_K \to \mathbb{R} \: | \: &\ell_i = \hat{\ell}_i \circ F_K^{-1}, \: \hat{\ell}_i \in \hat{\Sigma}_{\hat{K}}, \: 1 \leq i \leq n_K \}.
\end{align}}

The idea is that the unisolvence on the reference element can be ``transferred" to the physical elements.
Let $f \in P$, then notice that for $f$ restricted to the edge $[0,1]\times\{x_2=0\}$, $f$ is a third degree polynomial.
It can be seen that since $f$ is zero at two points and $f'$ is zero at two points, then $f$ will be identically zero on that edge.
Similarly, we can show that $f \equiv 0$ on the other three edges of our square.
This implies that our function has the following form,
\begin{equation*}
    f(x_1, x_2) = x_1(x_1 - 1) x_2(x_2 - 1) \gamma(x_1, x_2),
\end{equation*}
where $\gamma$ is a of the form $\gamma(x_1, x_2) = a + bx_1 + cx_2 + dx_1 x_2$.
Now consider the mixed partial derivative of $f$.
Using product rule, we have,
\begin{align*}
    f_{x_1 x_2} = (2x_1 - &1)(2x_2 - 1) \gamma + (2x_1 - 1) (x_2^2 - x_2) \gamma_{x_2} \\
    & + (x_1^2 - x_1)(2x_2 - 1) \gamma_{x_1} + (x_1^2 - x_1)(x_2^2 - x_2) \gamma_{x_1 x_2}.
\end{align*}
Note that $(2x_1 - 1)(2x_2 - 1) \neq 0$ for our vertices $\{a^i\}$, but the other three terms will vanish for each $a^i$.
Therefore we must have that $\gamma(a^i) = 0$ for $i = 1, 2, 3, 4$.
But by definition of $\gamma$, if $\gamma = 0$ for four points then $\gamma \equiv 0$. 
Thus the finite element is unisolvent.
$\blacksquare$


\vskip 2cm



\textbf{b.} What do you need to do to check if the above element with a rectangular mesh results in a $C^1$ finite element space?


\vskip 1cm


\textbf{Solution:} Let $p, q \in P$, with $p$ defined on $K_1$ and $q$ defined on $K_2$ where $K_1$ and $K_2$ are adjacent elements such that, $p$ and $q$ share the same values of the degrees of freedom on the interface between $K_1$ and $K_2$. 
Then we must show that the ``stiching together" of the functions $p$ and $q$ on the whole domain $K_1 \cup K_2$ results in a $C^1$ function.
Specifically, we require that the function,
\begin{equation}
	\phi(x,y) := \begin{cases}
		p(x,y) \quad &\text{if } (x,y) \in K_1, \\
		q(x,y) \quad &\text{if } (x,y) \in K_2,
	\end{cases}
\end{equation}
be a $C^1$ function.
$\blacksquare$


\vskip 2cm




{\bf c.} Does the above element (with a rectangular mesh) result in a $C^1$ finite element space? (Explain your answer).


\vskip 1cm

\textbf{Proof:} We follow the plan proposed in part b.
Let $a^1 = (x_1, y_1)$ and $a^2 = (x_1, y_2)$ be the points that form the line segment $\ell$; in particular, $\ell$ is a vertical line segment. 
Then we have,
\begin{align*}
	p|_\ell(a^1) = (q|_\ell)(a^1), &\qquad p|_\ell(a^2) = (q|_\ell)(a^2) \\
	(p_1)|_\ell(a^1) = (q_1)|_\ell(a^1), &\qquad (p_1)|_\ell(a^2) = (q_1)|_\ell(a^2) \\
	(p_2)|_\ell(a^1) = (q_2)|_\ell(a^1), &\qquad (p_2)|_\ell(a^2) = (q_2)|_\ell(a^2) \\
	(p_{12})|_\ell(a^1) = (q_{12})|_\ell(a^1), &\qquad (p_{12})|_\ell(a^2) = (q_{12})|_\ell(a^2)
\end{align*}

If we define,
\begin{align*}
    	f &= p|_\ell - q|_\ell, \\
    	g &= p_1|_\ell - q_1|_\ell.
\end{align*}
Then $f = f(y)$ is a cubic polynomial.
Note that $f(y_2) = f(y_1) = f'(y_2) = f'(y_1) = 0$ which implies that $f$ has 4 roots, hence $f \equiv 0$.
By a similar reasoning, we can conclude that $g \equiv 0$.
This implies that our function $\phi$ is continuous on $K_1 \cup K_2$.
In addition, the partial derivatives exist and are continuous on all of $K_1 \cup K_2$. 
These results come directly from the shared degrees of freedom.
Thus $\phi \in C^1(K_1 \cup K_2)$.
A similar argument can be used for a horizontal line segment. 
This shows that the above element will result in a $C^1$ finite element space.
$\blacksquare$


\vskip 2cm





\textbf{Problem 2.} Consider the Neumann Problem:
\begin{align} \label{pb2:eqs}
    -\Delta u &= f \quad \text{ in } \Omega \\
    \frac{\partial u}{\partial n} &= g \quad \text{ on } \partial \Omega.
\end{align}
Here $\Omega$ is a bounded domain in $\mathbb{R}^2$ and $f$ and $g$ are suitably smooth.

\vskip 1cm

\textbf{a.} Derive a weak form of the above problem using a test function in $H^1(\Omega)$.


\vskip 1cm

\textbf{Proof:} Multiplying $-\Delta u = f$ by a test function $v \in H^1(\Omega)$ and integrating over $\Omega$, we have,
\begin{align*}
	-\int_\Omega \Delta u v \: d\bs{x} &= \int_\Omega \nabla u \cdot \nabla v \: d\bs{x} - \int_{\partial \Omega} \frac{\partial u}{\partial n} v \: ds \\
	&= \int_\Omega \nabla u \cdot \nabla v \: d\bs{x} - \int_{\partial \Omega} g v \: ds.
\end{align*}
Adding the boundary term to the other side, we have,
\begin{equation*}
	a(u,v) := \int_\Omega \nabla u \cdot \nabla v \: d\bs{x} = \int_\Omega f v \: d\bs{x} + \int_{\partial \Omega} g v \: ds =: L(v).
\end{equation*}
So our weak formulation becomes: find $u\in H^1(\Omega)$ such that $a(u,v) = L(v)$ for all $v \in H^1(\Omega)$.
 



\vskip 2cm


\textbf{b.} Discuss when the weak form of Part a. has a solution and if it is unique.

\vskip 1cm

\textbf{Proof:} First note that $a$ and $F$ can both be shown to be continuous (for $F$ you need to invoke the trace lemma).
Then from the Lax-Milgram lemma a solution exists and is unique if $a$ is $H^1(\Omega)$-elliptic (coercive). 
However, in our case $H^1(\Omega)$-ellipticty will not hold since we do not have a Poincar\`{e} inequality (or any other means to bound $a(u,u)$ below).
Note that $v = \text{constant}$ is a permissible function in our test function space. 
So set $v = 1$ and we have
\begin{equation*}
	\int_\Omega f \: d\bs{x} + \int_{\partial \Omega} g \: ds = 0.
\end{equation*}
This is called the \textit{compatibility condition} or \textit{solvability condition} and is necessary for existance of the solution.

Next, note that if $u$ is a solution then $u + c$ is also a solution for $c \in \mathbb{R}$.
In order to guarantee a \textit{unique} solution we can impose the condition,
\begin{equation*}
	\int_\Omega u \: d\bs{x} = 0.
\end{equation*}
This assumption allows us to derive a Poincar\`{e} inequality, which allows us apply the Lax-Milgram lemma.
$\blacksquare$


\vskip 2cm




\textbf{c.} Describe a variational formulation of \eqref{pb2:eqs} in terms of an appropriate Hilbert space $V$. Be sure to explicitly define $V$.


\vskip 1cm


\textbf{Proof:} The variational formulationa will be: find $u \in V:= \{ v \in H^1(\Omega) \: : \: \int_\Omega v \: d\boldsymbol{x} = 0 \}$ such that $a(u,v) = L(v)$ for all $v \in V$.
Note we cannot impose both $\partial u/ \partial n = 0$ and $u = 0$ (Cauchy boundary condition) on $\partial \Omega$ since then the problem becomes ill-posed.

\vskip 2cm


\textbf{d.} Prove coercivity of the form of Part a. on the $V$ of Part c. when $\Omega = (0,1)^2$.



\vskip 1cm


\textbf{Proof:} In order to prove coercivity, we a Poincar\`{e} type inequality. 
Specifically, we will prove
\begin{equation*}
    ||u||^2_{L^2(\Omega)}  \leq C \Big[ \Big( \int_\Omega u \: dx \Big)^2 + ||\nabla u||^2_{L^2(\Omega)} \Big] = C ||\nabla u||^2_{L^2(\Omega)},
\end{equation*}
for some constant $C>0$. 
We will start by proving the result for a smooth function $\phi \in C^\infty(\Omega) \cap V$.
Note we have the following identity,
\begin{equation*}
    \phi(x_2, y_2) - \phi(x_1, y_1) = \int_{x_1}^{x_2} \frac{\partial \phi}{\partial x}(x,y_2) \: dx + \int_{y_1}^{y_2} \frac{\partial \phi}{\partial y}(x_1, y) \: dy.
\end{equation*}
Squaring both sides, we have,
\begin{align*}
    \phi^2(x_2, y_2) - 2\phi(x_2, y_2) \phi(x_1, y_1) + \phi^2(x_1, y_1)  &= \Big( \int_{x_1}^{x_2} \frac{\partial \phi}{\partial x}(x,y_2) \: dx + \int_{y_1}^{y_2} \frac{\partial \phi}{\partial y}(x_1, y) \: dy \Big)^2 \\
    &\leq 2 \Big( \int_{x_1}^{x_2} \frac{\partial \phi}{\partial x}(x,y_2) \: dx \Big)^2 + 2 \Big( \int_{y_1}^{y_2} \frac{\partial \phi}{\partial y}(x_1,y) \: dy \Big)^2 \\
    &\leq 2||\nabla \phi||^2_{L^2(\Omega)}.
\end{align*}
Now we integrate both sides of the inequality from 0 to 1 for $x_1$, $x_2$, $y_1$, and $y_2$ (four different variables).
Doing so will give us,
\begin{equation*}
    2||\phi||^2_{L^2(\Omega)} - 2\Big( \int_{\Omega} \phi \: dx \: dy \Big)^2 \leq 2 ||\nabla \phi \|^2_{L^2(\Omega)}.
\end{equation*}
Thus the result follows where $C = 1$.
Now to prove for $u \in H^1(\Omega)$, we use the fact that 
\begin{equation*}
    H^1(\Omega) = \overline{C^\infty(\Omega)}^{||\cdot||_{H^1(\Omega)}}.
\end{equation*}
So for $\varepsilon > 0$, we can find $\phi \in C^{\infty}(\Omega) \cap V$ such that $||u - \phi||^2_{H^1(\Omega)} < \varepsilon/16$. 
Now consider,
\begin{align*}
    ||u||^2_{L^2(\Omega)} &\leq ||u||^2_{H^1(\Omega)} \\
    &\leq (||u - \phi||_{H^1(\Omega)} + ||\phi||_{H^1(\Omega)})^2 \\
    &\leq 2||u - \phi||^2_{H^1(\Omega)} + 2||\phi||^2_{H^1(\Omega)} \\
    &\leq \frac{\varepsilon}{8} + 2||\phi||^2_{L^2(\Omega)} + 2 |\phi|^2_{H^1(\Omega)} \\
    &\leq \frac{\varepsilon}{8} + 4 |\phi|^2_{H^1(\Omega)} \\
    &\leq \frac{\varepsilon}{8} + 8|u - \phi|^2_{H^1(\Omega)} + 8 |u|^2_{H^1(\Omega)}.
\end{align*}
Then note that $|u - \phi|^2_{H^1(\Omega)} \leq ||u - \phi||^2_{H^1(\Omega)} < \varepsilon/16$. 
Thus we have,
\begin{equation*}
    ||u||^2_{L^2(\Omega)} \leq \frac{\varepsilon}{8} + \frac{\varepsilon}{2} + 8|u|^2_{H^1(\Omega)} < \varepsilon + 8|u|^2_{H^1(\Omega)}.
\end{equation*}
Since this holds for any $\varepsilon > 0$, we thus have $||u||_{L^2(\Omega)} \leq C ||\nabla u||_{H^1(\Omega)}$.
Since we have a Poincar\`{e} inequality, the coercivity proof follows in the typical way.
$\blacksquare$


\vskip 2cm


\textbf{Problem 3.} Let $\Omega_e = \{x \in \mathbb{R}^2 \: : \: ||x|| > 1 \}$. 
Show that the Poincar\`{e} inequality does not hold in $H^1_0(\Omega_e)$, i.e., there does not exist a constant $c > 0$ satisfying 
\begin{equation}
    c||u||^2_{L^2(\Omega_e)} \leq \int_{\Omega_e} ||\nabla u||^2 \: dx \quad \text{ for all } u \in H^1_0(\Omega_e).
\end{equation}
The space $H^1_0(\Omega_e)$ is the completion of $C^\infty_0(\Omega_e)$ in the norm 
\begin{equation}
    ||v||_{H^1(\Omega_e)} = \Big(||v||^2_{L^2(\Omega_e)} + ||\nabla v||^2_{(L^2(\Omega_e))^2} \Big)^{1/2}.
\end{equation}
(Hint: Consider dilating a fixed function.)


\vskip 1cm


\textbf{Proof:} Consider the rotationally symmetric bump function, 
\begin{equation*}
    \phi(r,\theta) := \phi(r) := \begin{cases}
        \exp(\frac{-1}{r(1-r)}) &\quad \text{ for } 0 < r < 1 \\
        0 &\quad \text{ for } r \geq 1,
    \end{cases}
\end{equation*}
which is only a function of $r$.
Consider the sequence of bump functions defined by,
\begin{equation*}
    \phi_n(r) := \phi\Big(\frac{r-1}{n}\Big).
\end{equation*}
Note that supp$(\phi_n) = [1,n+1]$.
Then consider the $L^2$ norm of $\phi_n$,
\begin{align*}
    ||\phi_n||^2_{L^2(\Omega_e)} &= \int_0^{2\pi} \int_1^\infty \phi^2_n(r) r \: dr \: d\theta \\
    &= 2\pi \int_1^{n+1} \phi^2(\frac{r-1}{n}) r \: dr \\
    &= 2\pi \int_0^1 \phi^2(s) n (ns + 1) \: ds \\
    &\geq 2\pi n^2 \int_0^1 s \phi^2(s) \: ds \\
    &= C_0n^2.
\end{align*}
Thus $\lim_{n\to\infty} ||\phi_n||_{L^2(\Omega_e)} = \infty$.
But consider the $H^1$-semi norm of $\phi_n$.
\begin{align*}
    |\phi_n|^2_{H^1(\Omega_e)} &= \int_0^{2\pi} \int_1^\infty |\phi'_n(r)|^2 r \: dr \: d\theta \\
    &= 2\pi \int_1^{n+1} \Big|\frac{\partial}{\partial r}\phi(\frac{r-1}{n})\Big|^2 r \: dr \\
    &= 2\pi \int_0^{1} \Big| \frac{1}{n} \frac{\partial}{\partial s} \phi(s) \Big|^2 n (ns + 1) \: ds \\
    &\leq 2\pi \frac{n(n+1)}{n^2} \int_0^1 |\phi'(s)|^2 \: ds \\
    &\leq C_1
\end{align*}
Thus since the $H^1$-semi norm is bounded for all $n$ and the $L^2$ norm blows up to infinity, this implies that we cannot have a Poincar\`{e} inequality on this domain.
$\blacksquare$



\end{document}
