\documentclass[11pt]{article}

\usepackage[english]{babel}
\usepackage[utf8]{inputenc}
\usepackage{amsmath}
\usepackage{amsfonts}
\usepackage[ampersand]{easylist}
\usepackage{authblk}
\usepackage{MnSymbol}
\usepackage{graphicx}
\usepackage[colorinlistoftodos]{todonotes}
\usepackage{geometry}
\usepackage{color}
\usepackage{mathtools}
\DeclarePairedDelimiter\ceil{\lceil}{\rceil}
\DeclarePairedDelimiter\floor{\lfloor}{\rfloor}

\geometry{letterpaper}
	\setlength{\oddsidemargin}{0cm}
	\setlength{\evensidemargin}{0cm}
	\setlength{\headheight}{0.5cm}
	\setlength{\headsep}{0cm}
	\setlength{\textwidth}{16cm}
	\setlength{\textheight}{21.0cm}
	\baselineskip=24pt

\newcommand{\wh}{\widehat}

\title{Applied/Numerical Qualifier Solution: August 2017}

\author{Bennett Clayton}
\affil{Texas A\&M University}
\date{\today}

\begin{document}
\maketitle

\textbf{Problem 1.} Let $K$ be a non-degenerate triangle in $R^2$.
Let $a_1$, $a_2$, $a_3$ be the three vertices of $K$.
Let $a_{ij} = a_{ji}$ denote the midpoint of the segment $(a_i,a_j)$, $i,j \in \{ 1, 2, 3\}$.
Let $\mathbb{P}^1$ be the set of linear functions $p(x_1,x_2)$ over $K$ and $\Sigma = \{\sigma_1, \sigma_2, \sigma_3 \}$ be the linear forms (or degrees of freedom)  on $\mathbb{P}^1$ defined as $\sigma_{ij}(p) = p(a_{ij})$, $i,j= 1,2,3$, $i \neq j$. 

\vskip 1cm


\textbf{a.} Show that the degrees of freedom $\{\sigma_{12},\sigma_{23}, \sigma_{31} \}$ are unisolvent.


\vskip 1cm


\textbf{Solution:} To prove unisolvence, we need to show that if $\sigma_{ij}(p) = 0$ for all $i \neq j$ and $p \in \mathbb{P}^1$, then $p \equiv 0$.
Consider $p$ in terms of the barycentric coordinates, that is, $p(x,y) = p(\lambda_1, \lambda_2, \lambda_3)$.
Now specifically, look at $p$ restricted to the line passing through the points $a_{12}$ and $a_{23}$, call it $L_2$.
Note that, in barycentric coordinates, $a_{12} = (1/2, 1/2, 0)$ and $a_{23} = (0, 1/2, 1/2)$.
Specifically, the line is defined as,
\begin{equation*}
    L_2 = \{ (\lambda_1, \lambda_2, \lambda_3) \in K \: : \: \lambda_2 = 1/2 \}.
\end{equation*}
Note that $p|_{L_2}$ is a 1-dimensional linear function that is zero at two points, we must have that $p|_{L_2} \equiv 0$.
Therefore, we can write $p$ as $p = a(\lambda_2 - 1/2)$.
Lastly, by assumption $p(1/2, 0, 1/2) = 0$, hence $-\frac{1}{2}a = 0$.
Therefore, $p \equiv 0$ and so $\Sigma$ is unisolvent.
$\blacksquare$



\vskip 2cm



\textbf{b.} Compute the “nodal” basis of $\mathbb{P}^1$ which corresponds to $\sigma = \{\sigma_{12},\sigma_{23}, \sigma_{31} \}$. 


\vskip 1cm


\textbf{Solution:} Using the same arguments as in part a. we can similarly construct the nodal basis of $\mathbb{P}^1$. 
So consider $\phi_2$ which satisfies, $\phi_2(a_{13}) = 1$, and $\phi_2(a_{23}) = \phi_2(a_{12}) = 0$.
Then $phi_2$ has the form, $\phi_2(\lambda_1, \lambda_2, \lambda_3) = a(\lambda_2 - \frac{1}{2})$.
So, since $\phi_2(\frac{1}{2}, 0, \frac{1}{2}) = 1$, we have that $a = -2$.
Hence $\phi_2 = -2(\lambda_2 - \frac{1}{2})$.
Repeating a similar argument for the other basis functions, we have,
\begin{align*}
    \phi_1 &= -2(\lambda_1 - \frac{1}{2}) \\
    \phi_2 &= -2(\lambda_2 - \frac{1}{2}) \\
    \phi_3 &= -2(\lambda_3 - \frac{1}{2}).
\end{align*}


\vskip 2cm


\textbf{c.} Let $\mathcal{T}_h$ be a triangulation of the domain $\Omega$ with polygonal boundary and let the finite dimensional space $V$ consist of functions whose restrictions to each $K$ are the functions from the FE $(K,\mathbb{P}^1,\Sigma)$.
Show that in general these functions are NOT in $H^1(\Omega)$.


\vskip 1cm


\textbf{Solution:} Consider the unite square formed by the two triangles,
\begin{align*}
    K_1 &= \text{conv}\{(0,0), (1,0), (0,1)\} \\
    K_2 &= \text{conv} \{ (1,0), (1,1), (0,1)\}.
\end{align*}
Then consider the functions $f_1(x,y) = x + y$ and $f_2(x,y) = 2x$ defined on $K_1$ and $K_2$ respectively.
Then note that $f_1(\frac{1}{2}, \frac{1}{2}) = 1 = f_2(\frac{1}{2}, \frac{1}{2})$.
But $f_1(1,0) = 1$ and $f_2(1,0) = 2$.
Therefore, a globally defined function on the whole square would not be continuous, even though the functions restricted to the elements are the same at the degrees of freedom.
Hence in general these functions are not in $H^1(\Omega)$.


\vskip 2cm


\textbf{d.}  If $M_K$ is the element “mass” matrix, evaluate its entries $m_{ij}$.


\vskip 1cm


\textbf{Solution:} By definition, the entries $m_{ij}$ are defined as,
\begin{equation*}
    m_{ij} = \int_{K} \phi_i \phi_j \: dA.
\end{equation*}
We can transform this integral to the reference triangle $\Hat{K}$ and then perform the integration, where $\lambda_1 = 1 - x - y$, $\lambda_2 = x$ and $\lambda_3 = y$. 
So consider,
\begin{align*}
    m_{11} &= \int_{K} (\phi_1)^2 \: dA \\
    &= \int_{\Hat{K}} (-2(\lambda_1 - \frac{1}{2}))^2 \frac{|K|}{|\Hat{K}|} \: dx \: dy \\
    &= 2|K| \int_0^1 \int_0^{1-x} 4(1 - x - y - \frac{1}{2})^2 \: dy \: dx \\
    &\quad \vdots \\
    &= \frac{1}{3}|K|.
\end{align*}
By symmetry arguments, $m_{22}$ and $m_{33}$ will be the same.
Similar computations will show that $m_{ij} = 0$ for $i\neq j$. 
Thus our mass matrix is,
\begin{equation*}
    M_K = \frac{1}{3}|K| \begin{bmatrix} 
        1 & 0 & 0 \\
        0 & 1 & 0 \\
        0 & 0 & 1
    \end{bmatrix}
\end{equation*}
$\blacksquare$




\vskip 2cm






{\bf Problem 2.}  
\vskip 1cm


\textbf{a.} Let $\Omega = (0,1)$. Assume that $u\in H^1(\Omega)$ and let $x_0 \in \overline{\Omega}$. Prove that 
\[ ||u||^2_{L^2(\Omega)} \leq C_1\left( u^2(x_0) + ||u'||^2_{L^2(\Omega)} \right) \]
with a constant $C_1$ independent of $x_0$. 


\vskip 1cm


\textbf{Proof:} We first prove the result for $\phi \in C^\infty(\overline{\Omega})$. 
Consider,
\begin{align*}
	||\phi||^2_{L^2(\Omega)} &= \int_0^1 \phi^2 \: dx \\ 
	&= \int_0^1 \Big( \int_{x_0}^x \phi'(t) \: dt + \phi(x_0) \Big)^2 \: dx \\
	&\leq \int_0^1 2 \Big( \int_{x_0}^x \phi'(t) \: dt \Big)^2 \: dx + 2 \phi(x_0)^2 \\
	&\leq 2 \int_0^1 |x - x_0| \Big|\int_{x_0}^x (\phi'(t))^2 \: dt \Big| \: dx + 2\phi(x_0)^2 \\ 
	&\leq 2(\phi^2(x_0) + |\phi|^2_{H^1(\Omega)})
\end{align*}
In the above work, we used the inequality $(a + b)^2 \leq 2a^2 + 2b^2$ and the Cauchy-Schwarz inequality. 

By the Meyers-Serrin theorem, for $u \in H^1(\Omega)$, we can find a sequence $\{ \phi_n \}_{n\in\mathbb{N}} \subset C^\infty(\Omega) \cap H^1(\Omega)$ such that $\phi_n \to u$ in $H^1(\Omega)$.
This implies that $||\phi_n||_{L^2(\Omega)} \to ||u||_{L^2(\Omega)}$ and $|\phi_n|_{H^1(\Omega)} \to |u|_{H^1(\Omega)}$.
Additionally, since $\Omega$ is 1D, functions in $H^1(\Omega)$ are continuous, so by the Sobolev embedding theorem, $||v||_{C^0(\overline{\Omega})} \leq C ||v||_{H^1(\Omega)}$.
This implies that $||\phi_n||_{C^0(\overline{\Omega})} \to ||u||_{C^0(\overline{\Omega})}$, i.e. $\phi_n(x_0) \to u(x_0)$. 
Thus, because we have convergence in $H^1$ and pointwise, the Poincar\`{e}-type inequality must hold for functions in $H^1(\Omega)$.
$\blacksquare$



\vskip 2cm



{\bf b.} Consider the fourth-order boundary value problem
\begin{equation}
    u^{(4)} = f \quad \text{ in } \Omega, \quad u(0) = 0, \quad u''(0) = 0, \quad  u''(1) + u'(1) = 1, \quad u'''(1) = 0.
\end{equation}
Derive a weak formulation of this problem assuming that $f \in L^2(\Omega)$.


\vskip 1cm


\textbf{Solution:} Multiply by a test function $v$ and integrate by parts,
\begin{align*}
    \int_0^1 u^{(4)} v \: dx &= u''' v \Big|_0^1 - \int_0^1 u''' v' \: dx \\
    &= u'''(1) v(1) - u'''(0) v(0) - u'' v'\Big|_0^1 + \int_0^1 u'' v'' \: dx \\
    &= -u'''(0) v(0) - u''(1) v'(1) + u''(0) v'(0) + \int_0^1 u'' v'' \: dx \\
	&= - u'''(0) v(0) -v'(1) + u'(1) v'(1) + \int_0^1 u'' v'' \: dx.
\end{align*}
Now since we have no information of $u'''(0)$, we will impose the condition, $v(0) = 0$, on our test functions.
So our bilinear and linear forms will be 
\begin{align*}
    a(u,v) &= \int_0^1 u'' v'' \: dx + u'(1) v'(1) \\
	f(v) &= \int_0^1 f v \: dx + v'(1).
\end{align*}
Lastly, our sobolev space will be,
\begin{equation*}
    V := \{ v \in H^2(0,1) \: : \: v(0) = 0 \},
\end{equation*}
and our weak formulation becomes: find $u \in V$ such that $a(u,v) = f(v)$ for all $v \in V$.
$\blacksquare$




\vskip 2cm


\textbf{c.} Show that the weak formulation that you derived in part b. above has a unique solution.


\vskip 1cm


\textbf{Proof:} We will use Lax-Milgram to show existence and uniqueness. 
First, it is an easy check to show that $f$ is continuous.
So lets show that $a$ is coercive. 
Note since our space is a subspace of $H^2$, our norm is $||\cdot ||_{H^2(0,1)}$.
We will need a trick from part a. that is, notice that if $u \in H^2(0,1)$ then $u' \in H^1(0,1)$. 
So the inequality from part a. can be applied to $u'$.
In addition, one can also prove the inequality,
\begin{equation*}
    ||u||^2_{L^2(0,1)} \leq |u|^2_{H^1(0,1)} \quad \text{ for all } u \in V
\end{equation*}
So consider,
\begin{align*}
    a(u,u) &= \int_0^1 (u'')^2 \: dx + (u'(1))^2 \\
    &\geq \frac{1}{2} |u|^2_{H^2(0,1)} + \frac{1}{2}(|u|^2_{H^2(0,1)} + (u'(1))^2) \\ 
    &\geq \frac{1}{2}|u|^2_{H^2(0,1)} + \frac{1}{4} |u|^2_{H^1(0,1)} \\
    &\geq \frac{1}{2}|u|^2_{H^2(0,1)} + \frac{1}{8}|u|^2_{H^1(0,1)} + \frac{1}{8}||u||^2_{L^2(0,1)}  \\
    &\geq \frac{1}{8} ||u||^2_{H^2(0,1)}.
\end{align*}
Therefore $a$ is coercive. 
Now for continuity, we have,
\begin{align*}
	|a(u,v)| &\leq \Big|\int_0^1 u''v'' \: dx\Big| + |u'(1) v'(1)| \\
	&\leq ||u''||_{L^2(0,1)} ||v''||_{L^2(0,1)} + |u'(1) v'(1)| \\
	&\leq ||u||_{H^2(0,1)} ||v||_{H^2(0,1)} + ||u'||_{C^0([0,1])} ||v'||_{C^0([0,1])} \\
	&\leq ||u||_{H^2(0,1)} ||v||_{H^2(0,1)} + C||u'||_{H^1(0,1)} ||v'||_{H^1(0,1)} \\
	&\leq C||u||_{H^2(0,1)} ||v||_{H^2(0,1)},
\end{align*}
and 
\begin{align*}
	|F(v)| &\leq \Big| \int_0^1 f(x) v(x) \: dx \Big| + |v'(1)| \\
	&\leq ||f||_{L^2(0,1)} ||v||_{L^2(0,1)} + C||v'||_{H^1(0,1)} \\
	&\leq (||f||_{L^2(0,1)} + C)||v||_{H^2(0,1)}. 
\end{align*}
Note we have not distinguished the constants in the proofs of continuity.
Thus by Lax-Milgram, we have existence and uniqueness.
$\blacksquare$



\vskip 2cm



\textbf{d.} Using Hermite cubic finite element spaces (i.e., piecewise cubic elements lying in $C^1(\Omega)$) derive a finite element method for the problem in part (b).
Be sure to carefully define your finite element space.

\vskip 1cm


\textbf{Proof:} Let $\widehat{K} = [0,1]$ be the reference element and $K_i := [x_{i-1},x_i]$ be the global elements for $i = 1, \ldots, N$, such that $\bigcup_{i=1}^N K_i = [0,1]$. 
In particular, we define our mesh as $\mathcal{T}_h := \{ K_i \}_{i=1}^N$.
We also define the affine geometric mappings, $T_{K_i} : \widehat{K} \to K_i$, defined as $T_{K_i}(\hat{x}) = h_i \hat{x} + x_{i-1}$ with $h_i := x_i - x_{i-1}$.
We now define our finite element space as,
\begin{equation}
	V_h := \{ v \in C^1(\Omega) : v\circ T_{K_i} \in \mathbb{P}_3, \forall K_i \in \mathcal{T}_h, v(0) = 0 \}.
\end{equation}
The basis for this space is composed of the cubic Hermite polynomials.
(Not going into the details about these polynomials here.)
Therefore, the discrete variational problem becomes: find $u_h \in V_h$ such that $a(u_h,v_h) = F(v_h)$ for all $v_h \in V_h$.
$\blacksquare$




\vskip 2cm

\textbf{e.} Show that the finite element method you derived has a unique solution $u_h$ and derive an optimal-order error estimate for $u - u_h$ in the $H^2(\Omega)$-norm. 
\textit{Hint:} A correct proof will involve using an interpolation error bound. You may state and use such a bound without proving it.

\vskip 0.5cm


\textbf{Proof}: First note that our finite dimensional space $V_h$ is a subspace of $V$, so by Lax-Milgram, we know there exists a unique solution $u_h\in V_h$.
Since we have a unique solution on our finite dimensional space, this means we can use Cea's Lemma.
In addition, let $\mathcal{I}_h$ be the interpolant of $u$ on our space $V_h$. Then we have the following,
\begin{align*}
	\Vert u - u_h\Vert_{H^2(\Omega)}^2 &\leq C \Big(\inf_{v_h\in V_h} \Vert u - v_h\Vert_{H^2(\Omega)}\Big)^2 \\
	&\leq C||u - \mathcal{I}_h(u)||_{H^2(\Omega)}^2 \\
	&= C\sum_{i=1}^N |u - \mathcal{I}_h(u)|^2_{H^2(K_i)} + |u- \mathcal{I}_h(u)|^2_{H^1(K_i)} + |u - \mathcal{I}_h(u)|^2_{L^2(K_i)}.
\end{align*}
Next, we transfer to the reference element.
\begin{align*}
	\Vert u - u_h\Vert_{H^2(\Omega)}^2 
	&\leq C \sum_{i=1}^N \Big( \int_{\wh{K}} \Big( \frac{1}{h^2}\frac{d^2}{d\wh{x}^2} (\wh{u} - \wh{\mathcal{I}}_h(\wh{u})) \Big)^2 h \: d\wh{x} 
	+ \int_{\wh{K}} \Big( \frac{1}{h}\frac{d^2}{d\wh{x}^2} (\wh{u} - \wh{\mathcal{I}}_h(\wh{u})) \Big)^2 h \: d\wh{x} \\
	&\qquad\qquad + \int_{\wh{K}} \big(\wh{u} - \wh{\mathcal{I}}_h(\wh{u}) \big)^2 h \: d\wh{x} \Big) \\
	&= C \sum_{i=1}^N \Big( \frac{1}{h^3} |(\text{Id} - \wh{I}_h)(\wh{u}) |^2_{H^2(\wh{K})}
		+ \frac{1}{h} |(\text{Id} - \wh{I}_h)(\wh{u}) |^2_{H^1(\wh{K})} 
	+  h \Vert(\text{Id} - \wh{I}_h)(\wh{u}) \Vert^2_{L^2(\wh{K})} \Big)
\end{align*}
Now we can use the Bramble-Hilbert lemma (or theorem) since 
$|(\text{Id} - \wh{\mathcal{I}}_h)(\cdot) |_{H^2(\wh{K})}$, 
$|(\text{Id} - \wh{\mathcal{I}}_h)(\cdot) |_{H^1(\wh{K})}$, and 
$\Vert(\text{Id} - \wh{\mathcal{I}}_h)(\cdot) \Vert_{L^2(\wh{K})}$ 
are bounded sublinear functionals which are all zero on $\mathbb{P}^3$.  
Applying the Bramble-Hilbert lemma and transferring back to the global element $K_i$ we have,
\begin{align*}
	\Vert u - u_h\Vert_{H^2(\Omega)}^2 &\leq C \sum_{i=1}^N \Big( \frac{1}{h^3}|\wh{u}|^2_{H^4(\wh{K})} 
+ \frac{1}{h}|\wh{u}|^2_{H^4(\wh{K})} + h|\wh{u}|^2_{H^4(\wh{K})} \Big) \\
	&\leq C\sum_{i=1}^N (h^{4-2})^2|u|_{H^4(K_i)}^2 + (h^{4-1})^2|u|_{H^4(K_i)}^2 + (h^4)^2\Vert u\Vert _{H^4(K_i)}^2 \\
	&\leq C\sum_{i=1}^N (h^4 + h^6 + h^8) |u|_{H^4(K_i)}^2 \\
	&\leq Ch^4\sum_{i=1}^N |u|_{H^4(K_i)}^2 \\
	&\leq Ch^4 |u|_{H^4(\Omega)}^2
\end{align*}
Taking the square root of both sides, we thus have our optimal error estimate
\[ ||u - u_h||_{H^2(\Omega)} \leq Ch^2 |u|_{H^4(\Omega)} \]
$\blacksquare$


\vskip 2cm




\textbf{Problem 3.} Let $u(x,t)$ be a smooth solution satisfying
\begin{equation}
    \partial_t u + \beta \partial_x u = 0, \quad x \in \Omega := (0,1), \quad t > 0 \quad \text{ and } \quad u(0,x) = \phi(x), \quad x\in \Omega
\end{equation}
where $\beta \in \mathbb{R}$ and $\phi$ is a given smooth function.
In addition, we assume that $u(x,t)$ satisfies the periodic boundary condition $u(0,t) = u(1,t)$, $t > 0$. 
Let $V = \{ v \in H^1(\Omega) : v(0) = v(1) \}$.


\vskip 1cm


\textbf{a.} Let $N\in \mathbb{N}$, set $h := \frac{1}{N+1}$ and consider the uniform mesh $\mathcal{T}_h$ composed of the cells $[x_i,x_{i+1}]$, $i= 0,\ldots,N$.
Let $P(\mathcal{T}_h)$ be the finite element space composed of continuous piecewise linear functions on $\mathcal{T}_h$.
Given $\phi_h \in V \cap P(\mathcal{T}_h)$ an approximation of $\phi$, consider the semi-discrete method: For $t >0$, find $u_h(t, \cdot) \in V \cap P(\mathcal{T}_h)$ such that $u_h(0,x) = \phi_h(x)$ and for every $v_h \in  P(\mathcal{T}_h)$ with $v_h(0) = v_h(1)$ there holds 
\begin{equation} \label{pb3:equation}
    \frac{h}{2} \sum_{i=0}^N (\partial_t u_h (t, x_{i+1}) v_h(x_{i+1}) + \partial_t u_h(t, x_i) v_h(x_i)) + \beta \int_\Omega \partial_x u_h(t,x) v_h(x) \: dx = 0.
\end{equation}
Show that the above problem can be reformulated as a system of ODEs and express this system in matrix-vector form. 

Note: we assume that as a function of $t$, $u_h(t) \to V \cap P(\mathcal{T}_h)$ is smooth.

\vskip 1cm

\textbf{Solution}: Let $\{\varphi_i\}_{i=0}^{N+1}$ be the basis of $P(\mathcal{T}_h)$ consisting of the usual ``tent" functions.
Then since $u_h(t,\cdot) \in V \cap P(\mathcal{T}_h)$ we write can write $u_h(t,x)$ as,
\begin{equation}
    u_h(t,x) = \sum_{j=0}^{N+1} U_j(t) \varphi_j(x).
\end{equation}
Then we can plug this $u_h$ into \eqref{pb3:equation} and take $v_h(x) = \varphi_k(x)$ for some $k \in \{0:N+1\}$.
Doing so gives us the following,
\begin{equation}
\begin{split}
    \frac{h}{2} \sum_{i=0}^N \Bigg[ \Big(\sum_{j=0}^{N+1} U_j'(t) \varphi_j(x_{i+1}) \Big) \varphi_k(x_{i+1}) &+ \Big(\sum_{j=0}^{N+1} U_j'(t)  \varphi_j(x_i) \Big) \varphi_k(x_i) \Bigg] \\
    &\qquad + \beta \sum_{j=0}^{N+1} U_j(t) \int_\Omega \varphi_j'(x) \varphi_k(x) \: dx = 0.
\end{split}
\end{equation}
Using the fact that $\varphi_j(x_i) = \delta_{j,i}$, we can further rewrite this equation as,
\begin{equation}
    \sum_{j=0}^{N+1} U'_j(t) \Big[ \frac{h}{2} \sum_{i=0}^N  \big( \delta_{j,i+1} \delta_{k,i+1} + \delta_{j,i} \delta_{k,i} \big) \Big] + \beta \sum_{j=0}^{N+1} U_j(t) \int_\Omega \varphi_j'(x) \varphi_k(x) \: dx = 0. 
\end{equation}
Let $M$ and $A$ be matrices with entries defined as, $m_{j,k} := \tfrac{h}{2} \sum_{i=0}^N (\delta_{j,i+1} \delta_{k,i+1} + \delta_{j,i} \delta_{k,i})$
and $a_{j,k} := \beta \int_{\Omega} \varphi'_j(x) \varphi_k(x) \: dx$, respectively.
Then we can write the problem as a system of ODEs in the following form,
\begin{equation}
    M\mathbf{U}'(t) + A\mathbf{U}(t) = 0,
\end{equation}
where $\mathbf{U}(t) = (U_0(t), \ldots, U_{N+1})$.
$\blacksquare$


\vskip 2cm


\textbf{b.} Show that the Finite Element approximation $u_h(t)$ satisfies
\begin{equation}
    \frac{d}{dt} \sum_{i=0}^N u_h(t,x_i)^2 = 0.
\end{equation}

\vskip 1cm

\textbf{Solution}: We start by setting $v_h := u_h(t,x)$ (for fixed $t$) in \eqref{pb3:equation}.
Then \eqref{pb3:equation} can be written as,
\begin{equation}
    \frac{h}{2} \sum_{i=0}^N \big(\tfrac{1}{2} \partial_t(u_h(t,x_{i+1})^2) + \tfrac{1}{2} \partial_t (u_h(t,x_i))^2 \big) + \beta \int_\Omega \tfrac{1}{2} \partial_x(u_h(t,x)^2) \: dx = 0.
\end{equation}
Note that the integral over omega will vanish due to the periodic boundary condition.
Then the remaining summation can be written in the following way,
\begin{equation}
    \frac{h}{2}  \frac{d}{dt} \Big[ \sum_{i=1}^{N} u_h(t,x_i)^2 + \tfrac{1}{2} u_h(t,x_0) + \tfrac{1}{2} u_h(t,x_{N+1})  \Big] = 0
\end{equation}
Again by the periodic boundary condition we have $u_h(t,x_{N+1}) = u_h(t,x_0)$. 
Thus we have, 
\begin{equation}
    \frac{d}{dt} \sum_{i=0}^N u_h(t,x_i)^2 = 0.
\end{equation}
$\blacksquare$

\vskip 2cm


\textbf{c.} Show that 
\begin{equation}
    c^{-1} \int_\Omega u^2_h(t,x) \: dx \leq h \sum_{i=0}^N u_h(t,x_i)^2 \leq c \int_\Omega u^2_h(t,x) \: dx
\end{equation}
and deduce the estimate 
\begin{equation}
    \int_\Omega u^2_h(t,x) \: dx \leq C\int_\Omega \phi^2_h(0,x) \: dx.
\end{equation}
Here $c$ and $C$ are constants independent of $h$.

\vskip 1cm

\textbf{Solution by Xin Su}: On the interval $[x_i,x_{i+1}], i=0,1,\cdots N$, $u_h(t,x)$ is a linear function and $u_h^2(t,x)$ is a concave up quadratic function. Hence, 

\begin{equation}
\label{eq:left}
\int_{x_i}^{x_{i+1}} u_h^2(t,x)\leq \frac{h}{2}(u_h^2(t,x_i)+u_h^2(t,x_{i+1}))
\end{equation}

On the other hand, $u_h(t,x)$ on $[x_i,x_{i+1}]$ can be written as $u_h(t,x)=u_h(t,x_{i}) + \frac{1}{h} (u_h(t,x_{i+1})-u_h(t,x_i)) (x-x_i)$.

\begin{equation}\label{eq:right}
\begin{aligned}
\int_{x_i}^{x_{i+1}} u_h^2(t,x) dx 
&= \Big(\frac{(u_h(t,x_{i+1})-u_h(t,x_{i}))^2}{3} + u_h(t,x_{i})u_h(t,x_{i+1})\Big) h \\
&= \frac{u_h^2(t,x_{i+1})+u_h(t,x_{i})u_h(t,x_{i+1})+u_h^2(t,x_{i})}{3}h \\
&\geq \frac{u_h^2(t,x_{i+1})-|u_h(t,x_{i})u_h(t,x_{i+1})|+u_h^2(t,x_{i})}{3}h \\ 
&\geq \frac{u_h^2(t,x_{i+1})+u_h^2(t,x_{i})}{6}h
\end{aligned}
\end{equation}


Summing up the equation \eqref{eq:left} and \eqref{eq:right} for $i=0,1,\cdots N$, we get 

\begin{equation}
	\frac{h}{6}(u_h^2(t,x_0)+2\sum_{i=1}^{N}u_h^2(t,x_{i})+ u_h^2(t,x_{N+1}))\leq \int_{0}^{1} u_h^2(t,x)\leq \frac{h}{2}(u_h^2(t,x_0)+2\sum_{i=1}^{N}u_h^2(t,x_{i})+ u_h^2(t,x_{N+1}))
\end{equation}

By the periodic boundary condition, we have $ u_h^2(t,x_{N+1})= u_h^2(t,x_{0})$. Thus, the desired inequality is proven with $c=3$.

%$$\frac{h}{3}(\sum_{i=0}^{N}u_h^2(t,x_{i}))\leq \int_{0}^{1} u_h^2(t,x)\leq h(\sum_{i=0}^{N}u_h^2(t,x_{i}))$$
\begin{equation}
\label{eq:ineq}
	\frac{1}{3}\int_{0}^{1} u_h^2(t,x ) \: dx \leq h\sum_{i=0}^{N}u_h^2(t,x_{i}) \leq 3\int_{0}^{1} u_h^2(t,x ) \: dx
\end{equation}
The above inequality holds for any fixed t>0. Use the inequality \eqref{eq:ineq} and combine the result of (b), we get 
\begin{equation}
	\int_{0}^{1} u_h^2(t,x ) \: dx \leq3 h\sum_{i=0}^{N}u_h^2(t,x_{i}) 
	= 3h\sum_{i=0}^{N}u_h^2(0,x_{i})\leq 9\int_{0}^{1} u_h^2(0,x ) \: dx 
	= 9\int_0^1 \phi_h^2(x) \: dx.
\end{equation}

\vskip 2cm



\end{document}

